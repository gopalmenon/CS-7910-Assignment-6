%%%%%%%%%%%%%%%%%%%%%%%%%%%%%%%%%%%%%%%%%
% Short Sectioned Assignment
% LaTeX Template
% Version 1.0 (5/5/12)
%
% This template has been downloaded from:
% http://www.LaTeXTemplates.com
%
% Original author:
% Frits Wenneker (http://www.howtotex.com)
%
% License:
% CC BY-NC-SA 3.0 (http://creativecommons.org/licenses/by-nc-sa/3.0/)
%
%%%%%%%%%%%%%%%%%%%%%%%%%%%%%%%%%%%%%%%%%

%----------------------------------------------------------------------------------------
%	PACKAGES AND OTHER DOCUMENT CONFIGURATIONS
%----------------------------------------------------------------------------------------

\documentclass[paper=a4, fontsize=11pt]{scrartcl} % A4 paper and 11pt font size

\usepackage[T1]{fontenc} % Use 8-bit encoding that has 256 glyphs
\usepackage{fourier} % Use the Adobe Utopia font for the document - comment this line to return to the LaTeX default
\usepackage[english]{babel} % English language/hyphenation
\usepackage{amsmath,amsfonts,amsthm} % Math packages

\usepackage{sectsty} % Allows customizing section commands
\usepackage[top=5em]{geometry}
\allsectionsfont{\centering \normalfont\scshape} % Make all sections centered, the default font and small caps

\usepackage{fancyhdr} % Custom headers and footers
\pagestyle{fancyplain} % Makes all pages in the document conform to the custom headers and footers
\fancyhead{} % No page header - if you want one, create it in the same way as the footers below
\fancyfoot[L]{} % Empty left footer
\fancyfoot[C]{} % Empty center footer
\fancyfoot[R]{\thepage} % Page numbering for right footer
\renewcommand{\headrulewidth}{0pt} % Remove header underlines
\renewcommand{\footrulewidth}{0pt} % Remove footer underlines
\setlength{\headheight}{5pt} % Customize the height of the header

\numberwithin{equation}{section} % Number equations within sections (i.e. 1.1, 1.2, 2.1, 2.2 instead of 1, 2, 3, 4)
\numberwithin{figure}{section} % Number figures within sections (i.e. 1.1, 1.2, 2.1, 2.2 instead of 1, 2, 3, 4)
\numberwithin{table}{section} % Number tables within sections (i.e. 1.1, 1.2, 2.1, 2.2 instead of 1, 2, 3, 4)

\setlength\parindent{0pt} % Removes all indentation from paragraphs - comment this line for an assignment with lots of text

\usepackage{mathtools}
\usepackage{amssymb}
\usepackage{gensymb}
\usepackage{chngcntr}
\usepackage{csquotes}
\usepackage{flexisym}
\usepackage{algorithm,algpseudocode}

\newcommand\Mycomb[2][n]{\prescript{#1\mkern-0.5mu}{}C_{#2}}

\counterwithout{figure}{section}
%----------------------------------------------------------------------------------------
%	TITLE SECTION
%----------------------------------------------------------------------------------------

\newcommand{\horrule}[1]{\rule{\linewidth}{#1}} % Create horizontal rule command with 1 argument of height

\title{	
\normalfont \normalsize 
\textsc{Utah State University, Computer Science Department} \\ [25pt] % Your university, school and/or department name(s)
\horrule{0.5pt} \\[0.4cm] % Thin top horizontal rule
\huge CS 7910 Computational Complexity\\Assignment 6 \\ % The assignment title
\horrule{2pt} \\[0.5cm] % Thick bottom horizontal rule
}

\author{Gopal Menon} % Your name

\date{\normalsize\today} % Today's date or a custom date

\begin{document}

\maketitle % Print the title

\begin{enumerate}
\item \textbf{(20 points)} In this exercise, we consider the \emph{Traveling Salesman Problem (TSP)}.

Let G be a \textbf{complete directed graph}. Each directed edge $(v_i,v_j)$ has a weight $w(v_i,v_j)$. It is possible that $w(v_i,v_j) \neq w(v_j,v_i)$, i.e., the edge weights can be asymmetric.

The optimization version of the TSP problem is to find a cycle $C$ in $G$ containing each vertex of $G$ exactly once such that the total weight of the edges of $C$ is minimized.

The decision problem of TSP is as follows. Given a value $W$, decide whether $G$ has a cycle $C$ that contains each vertex of $G$ exactly once such that the total weight of the edges of $C$ is at most $W$.

Prove that the decision problem of TSP is NP-Complete (Hint: by a reduction from the Hamiltonian Cycle problem).\\

The Hamiltonian Cycle problem is known to be NP-Complete. We can reduce the Hamiltonian Cycle problem to an instance of the TSP problem. Let $(G\textprime, W)$ be the instance of the TSP problem, where $W$ is an integer value equal to the number of edges in the Hamiltonian cycle and the graph $G\textprime$ is constructed from $G$ by 
\begin{enumerate}
\item creating a graph $G\textprime$ by incorporating every vertex and edge present in the Hamiltonian cycle of $G$
\item assigning a weight of $1$ both ways to every edge of $G\textprime$ 
\item creating an edge and assigning a weight of $\infty$ (infinity) to it between every node in $G\textprime$ that does not already have an edge
\end{enumerate} 

This reduction can clearly be done in polynomial time based on the following analysis. The first two steps of the reduction shown above can be done in $O(W)$ time. In the last step, we need to create edges having a weight of $\infty$ between all the node pairs in $G\textprime$ that do not already have edges. This can be done in time:

\begin{equation*}
\begin{aligned}
T &= \Mycomb[W-1]{2} -W\\
&= \frac{(W-1)!}{(W-3)!2!} - W\\
&= O(W^2)
\end{aligned}
\end{equation*}

So the reduction can be done in $O(W^2)$ time.

Given a certificate for the TSP problem, which will be an ordered list of edges, it is easy to verify in $O(N)$ time, where $N$ is the number of edges in certificate, or polynomial time, that the certificate is correct. This means that the TSP problem is in NP.

Consider the case where a Hamiltonian cycle exists in the graph $G$. In this case, for the TSP instance that was constructed, there exists a cycle that contains every node in it and has a total weight of edges of $W$.

In the case where the instance of the TSP problem is true (there exists a cycle in graph $G\textprime$ that has a total weight of edges equal to $W$), based on the way the graph $G\textprime$ was constructed, we know that there must exist a cycle in graph $G$. This means that we can reduce the Hamiltonian Cycle problem in polynomial time to the TSP problem. It follows that the TSP problem is in NP-Hard. Since we already know that the TSP problem is in NP, it follows that it is in NP-Complete.

\item This is a \enquote{warm-up} exercise on designing approximation algorithms.

A variation of the \emph{knapsack problem} is defined as follows (actually it is the subset-sum problem): Given as input a knapsack of size $K$ and $n$ items whose sizes are $k_1, k_2, \ldots , k_n$, where $K$ and $k_1, k_2, \ldots , k_n$ are all positive real numbers, find a full \enquote{packing} of the knapsack (i.e., choose a subset of the $n$ items such that the total sum of the sizes of the items in the chosen subset is \emph{exactly} equal to $K$).

We have proven that this problem is NP-complete, which implies that it is very likely that efficient algorithms (i.e., polynomial-time algorithms) for this problem do not exist. Thus, people tend to look for good \textbf{approximation algorithms} for solving this problem. In this exercise, we relax the constraint of the knapsack problem as follows. We still seek a packing of the knapsack, but we need not look for a \enquote{full} packing of the knapsack; instead, we look for a packing of the knapsack (i.e., a subset of the $n$ input items) such that the total sum of the sizes of the items in the chosen subset is at least $\frac{K}{2}$ (but no more than $K$). This is called a \emph{factor of} 2 \emph{approximation solution} for the knapsack problem. To simplify the problem, we assume that a factor of 2 approximation solution for the input always exists, i.e, there always exists a subset of items whose total size is at least $\frac{K}{2}$ and at most $K$.
\begin{enumerate}
\item \textbf{(20 points)} Design a \emph{polynomial-time} algorithm for computing a factor of 2 approximation solution, and analyze the running time of your algorithm (using the big-$O$ notation).
\item \textbf{(5 points)} Improve your algorithm to $O(n)$ time. If your algorithm for part (a) already runs in $O(n)$ time, then you will get the five points automatically.
\end{enumerate}
\textbf{Note:} You are required to clearly describe the main idea of your algorithm. Although the pseudo-code is not required, you may also give the pseudo-code if you feel it is helpful for you to explain your algorithm. You also need to briefly explain why your algorithm works, i.e., why your algorithm can produce a factor of 2 approximation solution. Finally, please analyze the running time of your algorithm.\\

If a factor 2 approximation solution is guaranteed to exist, we can go through the set of $n$ items from beginning to end and do the following. If any item is of size $ > K$, then we discard it. Any item found to be of size $ \ge \frac{K}{2} \le K$, can be returned as the solution. Else we can keep accumulating the other items (having size $< \frac{K}{2}$) and return the collection of such items as the required solution as soon as the sum of the collection crosses $\frac{K}{2}$. 

The reason this will work for the part where the items smaller than $\frac{K}{2}$ are accumulated is that, it is not possible to add an element of size $< \frac{K}{2}$ to the items being accumulated and have the total size exceed $K$. Consider the case where the total size is almost $\frac{K}{2}$. The largest item we can add to it can have a size that is almost $\frac{K}{2}$, but will have to be $< \frac{K}{2}$. By adding this item, the total size will be almost $K$, but will still be $< K$. And as soon as the total size crosses $\frac{K}{2}$, we will terminate the process.

The algorithm is shown in Algorithm listing \ref{Fac2ApproxAlg} where it receives the set of items $I$ and the knapsack size $K$ as parameters.

\begin{minipage}{\linewidth}
  \begin{algorithm}[H]
    \caption{Factor 2 Approximation Algorithm}\label{Fac2ApproxAlg}
    \begin{algorithmic}[1]
      \Procedure{Factor 2 Approximation Algorithm}{$I, K$}
        \State create $array\_of\_items$ of size $I.length$
        \State $array\_sum = 0$
	\For{all $k_i$ in $I$}
		\If {$k_i \ge \frac{K}{2} \le K$}
		  \State return $k_i$
		\ElsIf {$k_i< \frac{K}{2}$} 
		  \State insert $k_i$ into $array\_of\_items$
		  \State $array\_sum = array\_sum + k_i$
		  \If {array_sum $\ge \frac{K}{2}$}
		    \State return $array\_of\_items$
		  \EndIf
		\EndIf 
	\EndFor
      \EndProcedure
    \end{algorithmic}
  \end{algorithm}
\end{minipage}\\

Since the factor 2 approximation algorithm is able to achieve its objective by going through all the items in one pass, the running time will be $O(n)$, where $n$ is the number of items.

\end{enumerate}

%----------------------------------------------------------------------------------------

\end{document}